\documentclass[]{article}
\usepackage{lmodern}
\usepackage{amssymb,amsmath}
\usepackage{ifxetex,ifluatex}
\usepackage{fixltx2e} % provides \textsubscript
\ifnum 0\ifxetex 1\fi\ifluatex 1\fi=0 % if pdftex
  \usepackage[T1]{fontenc}
  \usepackage[utf8]{inputenc}
\else % if luatex or xelatex
  \ifxetex
    \usepackage{mathspec}
  \else
    \usepackage{fontspec}
  \fi
  \defaultfontfeatures{Ligatures=TeX,Scale=MatchLowercase}
\fi
% use upquote if available, for straight quotes in verbatim environments
\IfFileExists{upquote.sty}{\usepackage{upquote}}{}
% use microtype if available
\IfFileExists{microtype.sty}{%
\usepackage{microtype}
\UseMicrotypeSet[protrusion]{basicmath} % disable protrusion for tt fonts
}{}
\usepackage{hyperref}
\hypersetup{unicode=true,
            pdfborder={0 0 0},
            breaklinks=true}
\urlstyle{same}  % don't use monospace font for urls
\IfFileExists{parskip.sty}{%
\usepackage{parskip}
}{% else
\setlength{\parindent}{0pt}
\setlength{\parskip}{6pt plus 2pt minus 1pt}
}
\setlength{\emergencystretch}{3em}  % prevent overfull lines
\providecommand{\tightlist}{%
  \setlength{\itemsep}{0pt}\setlength{\parskip}{0pt}}
\setcounter{secnumdepth}{0}
% Redefines (sub)paragraphs to behave more like sections
\ifx\paragraph\undefined\else
\let\oldparagraph\paragraph
\renewcommand{\paragraph}[1]{\oldparagraph{#1}\mbox{}}
\fi
\ifx\subparagraph\undefined\else
\let\oldsubparagraph\subparagraph
\renewcommand{\subparagraph}[1]{\oldsubparagraph{#1}\mbox{}}
\fi

\date{}

\begin{document}

\subsection{Synopsis}\label{synopsis}

This project is a prototype for an online database to collect
information about cows and farms. It relies on the MEAN stack.

This means that it requires Node.js, which can be downloaded from:
https://nodejs.org/en/download/

\subsection{Installation}\label{installation}

To use: - open a terminal / command prompt, and \texttt{cd} to a
directory. - type in the command
\texttt{git\ clone\ https://github.com/dhruuuuuv/amrapp.git} - change
into the new directory with \texttt{cd\ amrapp} - ensure node is
installed by typing \texttt{node\ -v}. If it is a version number should
be displayed. - enter the command \texttt{npm\ start}, and you should be
greeted with: \texttt{\textgreater{}\ node\ ./bin/www} - in your
web-browser, type in the address bar \texttt{http://localhost:3000},
after a while you should be greeted with the cowllection homepage!

\subsection{Usage}\label{usage}

Navigate through the app by clicking on the relevant \textbf{farm
number}. This should then reveal a \textbf{farm overview} with multiple
\textbf{animal id}s.

If you click on an animal id, the information about that ID is revealed
below, so \textbf{scroll down} to reveal the new information about the
specfic cow.

Each cow has associated isolates, which also by clicking on the relevant
\textbf{isolate number} will reveal the specific data about that
isolate.

At any point, click on another \textbf{animal id} to view information
about that, or another \textbf{isolate number} for that. Click on the
HOME button to return to the farm overview.

\end{document}
